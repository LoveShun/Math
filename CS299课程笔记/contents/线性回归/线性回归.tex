\section{线性回归}
\subsection{假设函数(Hypothesis Function)}
\begin{equation}\begin{aligned}
	h_\theta(x) = \theta_0 + \theta_1x_1 + \theta_2x_2 + \theta_3x_3 + ... + \theta_nx_n
\end{aligned}\end{equation}
为每个数据点添加$x_0=1$,则
\begin{equation}\begin{aligned}
	h_\theta(x) &= \theta_0x_0 + \theta_1x_1 + \theta_2x_2 + \theta_3x_3 + ... + \theta_nx_n \\
	&= \sum_{j=0}^n{\theta_jx_j}
\end{aligned}\end{equation}

\subsection{最小二乘法}
后面要讲到的Cost Function使用的就是最小二乘法

\subsection{梯度下降}
\subsubsection{批梯度下降(BGD)}
\begin{enumerate}
	\item Cost Function
	\begin{equation}\begin{aligned}
		J(\theta) = \frac{1}{2} \sum_{i=1}^m \left[h_{\theta} {(x^{(i)})} - y^{(i)}\right]^2
	\end{aligned}\end{equation}
	其中,$J(\theta)$就是在训练过程中的Cost Function。我们的目的就是将该Cost Function最小化。\\
	上式中:
	\begin{itemize}
		\item $x^{(i)}$为输入,是个$n$维列向量;
		\item $y^{(i)}$为输出,是一个标量;
		\item $\theta$为所要训练的参数,也是一个$n$维列向量;
		\item $h_{\theta}$为假设函数,我们通过它来预测$y^{(i)}$的值;
		\item $h_{\theta} {(x^{(i)})}$为输入经过假设函数后得到的输出,是$y^{(i)}$的点估计量
	\end{itemize}

	\item 迭代方式
	\begin{equation}\begin{aligned}
	      \theta_j &:= \theta_j - \alpha \frac{\partial} {\partial \theta_j} J(\theta)
	\end{aligned}\end{equation}
	其中:
	\begin{equation}\begin{aligned}
	      \frac{\partial} {\partial \theta_j} J(\theta) &= \frac{\partial}{\partial \theta_j} \frac{1}{2} \sum_{i=1}^m\left[ h_\theta(x^{(i)}) - y^{(i)} \right]^2 \\
	      &= 2 * \frac{1}{2} \sum_{i=1}^m\left[ h_\theta(x^{(i)}) - y^{(i)} \right] \frac{\partial}{\partial\theta_j}\left[ h_\theta(x^{(i)}) - y^{(i)} \right] \\
	      &= \sum_{i=1}^m\left[ h_\theta(x^{(i)}) - y^{(i)} \right]\frac{\partial}{\partial\theta_j}\left[ \theta_0x_0^{(i)} +  \theta_1x_1^{(i)} + \dots + \theta_jx_j^{(i)} + ... + \theta_nx_n^{(i)} - y^{(i)} \right] \\
	      &= \sum_{i=1}^m\left[ h_\theta(x^{(i)}) - y^{(i)} \right]x_j^{(i)}
	\end{aligned}\end{equation}
	所以,更新后的梯度下降公式为
	\begin{equation}\begin{aligned}
		\theta_j &:= \theta_j - \alpha \sum_{i=1}^m \left[ h_\theta(x^{(i)}) - y^{(i)} \right]x_j^{(i)}
	\end{aligned}\end{equation}
	其中,$\alpha$称为学习速率(learning rate),用于控制$\theta$前进的步伐,避免$\theta$走得太快(或太慢)\footnote{走得太慢需要的迭代很多次数才能收敛,走得太快有可能出现Cost Function值越来越大的情况,即Cost随着迭代次数增多不但不收敛反倒一直增大。\\
	在实践时,时常会画出Cost随迭代次数变化的图像(设置一定的次数间隔,不会每次都画),以实时观察学习效果}。\\
	梯度下降法每次计算都是找到当前所在位置中,下降最快的方向\footnote{至于为什么是当前所在位置中下降最快的方向就是梯度的定义了。不明白的请复习高数}。 \\
	如果计算进入了某个局部极小值,则很可能一直在这个局部极小值内出不来了\footnote{因为在这局部极小值中,各个方向的导数$\frac{\partial{J(\theta)}}{\theta_j}$都很小,其和也很小,故$\theta$一直徘徊在这局部极小值附近},除非你的步长比较大,帮助它跑出这个局部极小值\footnote{实际上,我们并不知道现在处在的是局部最小还是全局最小,为了避免Cost随迭代次数变大,我们也不会将步长设得太大},这就是梯度下降法的局限所在,它很容易陷入局部极小值中。 \\
	使用梯度下降法每次修正的是参数$\theta$,而不是输入的$x$或$y$\footnote{吴恩达在Coursera的课程中会画图帮助理解$\theta$的迭代过程,要注意,图像中的横轴是$\theta_j$,而不是$x_j$}。
\end{enumerate}


\subsubsection{随机梯度下降(SGD)}
由批梯度下降的式子可以发现,在每一次梯度下降的迭代过程中,我们都遍历了所有训练集,这将会耗费大量的性能,于是产生了随机梯度下降(SGD)方法,每次迭代只使用一个数据\footnote{因为每次只使用1个数据收敛得太慢了,因此一般不会使用此方法,大部分情况下会使用下面提到的迷你批梯度下降}。 \\
\begin{enumerate}
	\item Cost Function
	\begin{equation}\begin{aligned}
		J(\theta) = \frac{1}{2} \left[h_{\theta} {(x^{(i)})} - y^{(i)}\right]^2
	\end{aligned}\end{equation}

	\item 迭代方式
	\begin{equation}\begin{aligned}
	      \frac{\partial} {\partial \theta_j} J(\theta) &= \left[ h_\theta(x^{(i)}) - y^{(i)} \right]x_j^{(i)}
	\end{aligned}\end{equation}
	更新后
	\begin{equation}\begin{aligned}
		\theta_j &:= \theta_j - \alpha\left[ h_\theta(x^{(i)}) - y^{(i)} \right]x_j^{(i)}
	\end{aligned}\end{equation}
	如上所示,与批梯度下降不同,随机梯度下降每次迭代只用了一个数据(第$i$个数据),若$i$从1取到m,则完成了一次训练集的遍历\footnote{在训练中,为了使参数收敛,可以多次遍历训练集。比如我们的批梯度下降就每次都遍历了一遍训练集,随机梯度(或其他梯度下降方法)当然也可以多次遍历}。\\
	使用随机梯度下降虽然解决了批梯度下降耗费过多性能的问题,但是却带来了另一个问题:收敛太慢!由此,我们折中使用迷你批梯度下降(mini-batch GD)。\\
\end{enumerate}

\subsubsection{迷你批梯度下降(mini-batch GD)}
为了解决梯度下降每次都使用所有训练集导致的性能问题,以及随机梯度下降每次只使用一个数据导致的收敛太慢,我们可以只用mini-batch GD。每次只用一部分训练集进行训练。
\begin{enumerate}
	\item Cost Function \\
	略\footnote{与批梯度下降相比,只是求和的上限变了而已,从训练集大小$m$变成了一个比较合理的数字}

	\item 迭代方式 \\
	略
	\item 在很多第三方框架中使用的都是迷你梯度下降,故其在使用过程中都需要设置每次迭代使用的数据量大小。
\end{enumerate}

\subsubsection{其他}
\begin{enumerate}
	\item 在线性回归中,使用特征缩放可提高收敛速度
\end{enumerate}







\section{随机变量的数字特征}
\subsection{随机变量的数学期望和方差}
\subsubsection{期望}
\begin{enumerate}
	\item 数学期望
	\begin{enumerate}
		\item 离散型
			\begin{equation}
				E(X) = \sum_{k=1}^{+\infty}x_kp_k
			\end{equation}
			如果上式绝对收敛,则期望存在;否则期望不存在
		\item 连续型
			\begin{equation}
				E(X) = \int_{-\infty}^{+\infty}xf(x)dx
			\end{equation}
			同样地,若上式绝对收敛,则期望存在;否则不存在
	\end{enumerate}

	\item 数学期望的性质
	\begin{enumerate}
		\item 
		\begin{equation}
			E(CX) = CE(X)
		\end{equation}
		\item 
		\begin{equation}
			E(X \pm Y) = E(X) \pm E(Y)
		\end{equation}
		\item 若随机变量$X$, $Y$相互独立,则
		\begin{equation}
			E(XY) = E(X)E(Y)
		\end{equation}
		其实,$X, Y$不相关即可,不相关就是上式成立的充要条件
	\end{enumerate}

	\item $Y=g(X)$的数学期望
	\begin{enumerate}
		\item 离散型
			\begin{equation}
				E(Y) = E\left(g(X)\right) = \sum_{k=1}^{+\infty} g(x_k)p_k
			\end{equation}
			如果上式绝对收敛,则期望存在;否则期望不存在
		\item 连续型
			\begin{equation}
				E(Y) = E(g(X)) = \int_{-\infty}^{+\infty}g(x)f(x)dx
			\end{equation}
			同样地,若上式绝对收敛,则期望存在;否则不存在
	\end{enumerate}

	\item $Z=g(X,Y)$的数学期望
	\begin{enumerate}
		\item 离散型
			\begin{equation}
				E(Y) = E(g(X,Y)) = \sum_{i=1}^{+\infty}\sum_{j=1}^{+\infty} g(x_i, y_j)p_{ij}
			\end{equation}
			如果上式绝对收敛,则期望存在;否则期望不存在
		\item 连续型
			\begin{equation}
				E(Y) = E(g(X,Y)) = \int_{-\infty}^{+\infty}\int_{-\infty}^{+\infty}g(x,y)f(x,y)dxdy
			\end{equation}
			同样地,若上式绝对收敛,则期望存在;否则不存在
	\end{enumerate}
\end{enumerate}

\subsubsection{方差}
\begin{enumerate}
	\item 方差
	\begin{enumerate}
		\item 方差的定义
		\begin{equation}
			D(X) = E\{\left[X-E(X)\right]^2\}
		\end{equation}
		记做$\sigma^2$
		\item 方差计算公式
		\begin{equation}
			D(X) = E(X^2) - \left[ E(X) \right]^2
		\end{equation}
		另,因为方差恒$\geq 0$,故由上式可得出$E(X^2) \geq \left[ E(X) \right]^2$
	\end{enumerate}

	\item 方差的性质
	\begin{enumerate}
		\item 
		\begin{equation}
			D(aX+b) = a^2D(X)
		\end{equation}
		\item $X,Y$相互独立时
		\begin{equation}
			D(X\pm Y) = D(X) + D(Y)
		\end{equation}
		注意,不论是$X+Y$还是$X-Y$,结果均是方差之和
	\end{enumerate}
\end{enumerate}

\subsubsection{常用随机变量的数学期望和方差}
\begin{enumerate}
	\item $0-1$分布
	\begin{align}
		E(X) &= p \\
		D(X) &= p(1-p)
	\end{align}
	\item 二项分布: $X \sim B(n,p)$
	\begin{align}
		E(X) &= np \\
		D(X) &= np(1-p)
	\end{align}
	\item 泊松分布: $X \sim P(\lambda)$
	\begin{align}
		E(X) &= \lambda \\
		D(X) &= \lambda
	\end{align}
	\item 几何分布: $P\{X=k\} = p(1-p)^{k-1}, \quad k = 1, 2, \dots, \quad 0<p<1$
	\begin{align}
		E(X) &= \frac{1}{p} \\
		D(X) &= \frac{1-p}{p^2}
	\end{align}
	\item 均匀分布: $X \sim U(a,b)$
	\begin{align}
		E(X) &= \frac{a+b}{2} \\
		D(X) &= \frac{(b-a)^2}{12}
	\end{align}
	\item 指数分布: $X \sim E(\lambda)$
	\begin{align}
		E(X) &= \frac{1}{\lambda} \\
		D(X) &= \frac{1}{\lambda^2}
	\end{align}
	\item 正态分布: $X \sim N(\mu, \sigma^2)$
	\begin{align}
		E(X) &= \mu \\
		D(X) &= \sigma^2
	\end{align}
\end{enumerate}

\subsection{矩、协方差、相关系数}
\begin{enumerate}
	\item 矩
	\begin{enumerate}
		\item $X$的$k$阶原点矩
		\begin{equation}
			E(X^k), \quad k = 1, 2, \dots
		\end{equation}

		\item $X$的$k$阶中心矩
		\begin{equation}
			E\left\{\left[X-E(X) \right]^k\right\}, \quad k,l = 1, 2, \dots
		\end{equation}

		\item $X,Y$的$k+l$阶原点矩
		\begin{equation}
			E(X^kY^l), \quad k = 1, 2, \dots
		\end{equation}

		\item $X$的$k$阶中心矩
		\begin{equation}
			E\left\{\left[X-E(X) \right]^k \left[Y-E(Y) \right]^l\right\}, \quad k,l = 1, 2, \dots
		\end{equation}
	\end{enumerate}
	
	\item 协方差 
	\footnote{协方差在大学数学中用到的不多,但是在机器学习中时常用到,尤其是多重高斯分布时} \\
	\begin{enumerate}
		\item 如果$E\left\{\left[X-E(X)\right]\left[Y-E(Y)\right]\right\}$存在,则$X,Y$的协方差为:
		\begin{align}
			\mathrm{cov}(X,Y) = E\left\{\left[X-E(X)\right]\left[Y-E(Y)\right]\right\}
		\end{align}
		\footnote{方差可看成$X=Y$时的特殊情况} \\
		可理解成两个方向上当前值与均值间的差异组成的矩形面积的期望
		\item 
		\begin{equation}
			\mathrm{cov}(X,Y) = E(XY) - E(X)E(Y)
		\end{equation}

		\item 
		\begin{equation}
			D(X\pm Y) = D(X) + D(Y) \pm 2\mathrm{cov}(X,Y)
		\end{equation}

		\item 
		\begin{equation}
			\mathrm{cov}(X,Y) = \mathrm{cov}(Y,X)
		\end{equation}

		\item 
		\begin{equation}
			\mathrm{cov}(aX, bY) = ab\mathrm{cov}(X,Y), \quad a,b\in {\rm I\!R}
		\end{equation}

		\item 
		\begin{equation}
			\mathrm{cov}(X_1+X_2, Y) = \mathrm{cov}(X_1, Y) + \mathrm{cov}(X_2, Y)
		\end{equation}
	\end{enumerate}

	\item 协方差矩阵 \\
	\begin{enumerate}
		\item 定义 \\
		假设$X$是以$n$个标量随机变量组成的列向量,且$\mu_k$是其第$k$个元素的期望值,即$\mu_k=E(X_k)$,则协方差矩阵被定义为:
		\begin{equation}
			\Sigma = E\left\{\left[X-E(X)\right]\left[X-E(X)\right]^T\right\}
		\end{equation}
		矩阵中的第$(i,j)$个元素是$x_i$与$x_j$的协方差。
		\begin{equation}
		\left[\begin{matrix}
		E\left[(X_1-\mu_1)(X_1-\mu_1)\right] & E\left[(X_1-\mu_1)(X_2-\mu_2)\right] & \dots & E\left[(X_1-\mu_1)(X_n-\mu_n)\right] \\
		E\left[(X_2-\mu_2)(X_1-\mu_1)\right] & E\left[(X_2-\mu_2)(X_2-\mu_2)\right] & \dots & E\left[(X_2-\mu_2)(X_n-\mu_n)\right] \\
		\vdots & \vdots & \ddots & \vdots \\
		E\left[(X_n-\mu_n)(X_1-\mu_1)\right] & E\left[(X_n-\mu_n)(X_2-\mu_2)\right] & \dots & E\left[(X_n-\mu_n)(X_n-\mu_n)\right] \\
		\end{matrix}\right]
		\end{equation}
	\end{enumerate}
	
	
	\item 相关系数
	\begin{enumerate}
		\item 
		\[ \rho_{XY}=\begin{cases}
			\frac{\mathrm{cov}(X,Y)}{\sqrt{D(X)}\sqrt{D(Y)}}, \quad \sqrt{D(X)}\sqrt{D(Y)} \neq 0 \\
			0, \quad \sqrt{D(X)}\sqrt{D(Y)} = 0
		\end{cases} \]
		由上式可知,$\rho_{XY} = 0$有两种可能,$\sqrt{D(X)}\sqrt{D(Y)} = 0$或$\mathrm{cov}(X,Y)=0$

		\item 如果$\rho_{XY}=0$,则称$X$与$Y$不相关

		\item 
		\begin{equation}
			\left|\rho_{XY}\right| \leq 1
		\end{equation}
		\item $\rho_{XY} = 1$的充分必要条件是:存在不全为$0$的常数$a,b$,使得
		\begin{equation}
			P(aX+bY=1) = 1
		\end{equation}
	\end{enumerate}

	\item 独立与不相关
	\begin{enumerate}
		\item 若随机变量$X,Y$相互独立,则$X,Y$必不相关;反之却不成立。
		\item 对于二维正态随机变量$(X,Y)$,其相互独立与其不相关是等价的,即:$\rho=0$是$X,Y$相互独立的充分必要条件。\\
		但若$(X,Y)$不是二维正态随机变量,则没有这个性质。
	\end{enumerate}
	

\end{enumerate}


















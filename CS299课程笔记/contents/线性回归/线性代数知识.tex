\subsection{线性代数知识}\footnote{在实际的机器学习实践中,都是使用矩阵运算,以提高性能}
\begin{enumerate}
	\item
	\begin{equation}
		\nabla_{\theta}J(\theta) = \left[\begin{matrix}
		\frac{\partial J}{\partial\theta_0} \\
		\frac{\partial J}{\partial\theta_1} \\
		\vdots \\
		\frac{\partial J}{\partial\theta_n} \\
		\end{matrix}\right], \quad \in {\rm I\!R}^{n+1}
	\end{equation}

	\item
	\begin{equation}
		\nabla_Af(A) = \left[ \begin{matrix}
			\frac{\partial f}{\partial A_{11}} & \frac{\partial f}{\partial A_{12}} & \dots & \frac{\partial f}{\partial A_{1n}} \\
			\frac{\partial f}{\partial A_{21}} & \frac{\partial f}{\partial A_{22}} & \dots & \frac{\partial f}{\partial A_{2n}} \\
			\vdots & \vdots & \ddots & \vdots \\
			\frac{\partial f}{\partial A_{n1}} & \frac{\partial f}{\partial A_{n2}}& \dots & \frac{\partial f}{\partial A_{nn}} \\
		\end{matrix}\right]
	\end{equation}

	\item 矩阵的迹的计算方式 \\
	如果矩阵$A$是方阵,则:
	\begin{equation}
		tr(A) = \sum_{i=1}^nA_{ii}
	\end{equation}

	\item 矩阵的迹的性质
	\begin{align}
		tr(AB) &= tr(BA) \\
		tr(ABC) &= tr(CAB) = tr(BCA) \\
		tr(A^T) &= tr(A) \\
		tr(A+B) &= tr(A) + tr(B) \\
		tr(a\cdot A) &= a\cdot tr(A)
	\end{align}

	\item 
	\begin{align}
		\nabla_Atr(AB) &= B^T \\
		\nabla_{A^T}f(A) &= \left[\nabla_Af(A)\right]^T \\
		\nabla_Atr(ABA^TC) &= CAB + C^TAB^T \\
		\nabla_A|A| &= |A|(A^{-1})^T \\
		A^{-1} &= \frac{(A^{'})^T}{|A|}
	\end{align}
	其中,$A^{-1}$为矩阵的逆



\end{enumerate}



















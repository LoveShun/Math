\section{广义线性模型}
\subsection{指数分布族}
\begin{enumerate}
	\item 指数分布族的一般形式
	\begin{equation}
		p(y;\eta) = b(y)e^{\eta^T T(y) - a(\eta)}
	\end{equation}
	其中,$\eta$称为自然参数(Natural Parameter); \\
	$T(y)$为$\eta$\footnote{\color{red}{应该是$\eta$,因为指数分布族的参数只有$\eta$,但正确与否待确认}}的充分统计量(Sufficient Statistic),大部分情况下,$T(y) = y$;在$T(y)=y$时,$\eta$往往为标量; \\
	$a(\eta)$称为 log partition function;
	\item 
\end{enumerate}

\subsection{伯努利分布与高斯分布中,GLM各部分的值}
主要是将伯努利分布(即$0-1$分布)与高斯分布的式子尽可能地转化成指数分布族的一般形式,然后在根据该一般形式写出$\eta$,$T(y)$,$a(\eta)$的形式。\\
\begin{enumerate}
	\item 伯努利分布
	\begin{align}
		P(y;\phi) &= \phi^y(1-\phi)^{1-y} \\
		&= e^{y\ln \phi + (1-y)\ln(1-\phi)} \\
		&= e^{ln{\frac{\phi}{1-\phi}} \cdot y + \ln(1-\phi)}
	\end{align}
	与广义线性模型的一般形式比较,可以发现:
	\begin{align}
		b(y) &= 1 \\
		\eta^T &= \ln{\frac{\phi}{1-\phi}} = \eta \rightarrow \phi = \frac{e^{\eta}}{1+e^{\eta}} \\
		T(y) &= y \\
		a(\eta) &= -\ln(1-\phi) = -\ln{(1-\frac{e^{\eta}}{1+e^{\eta}})} = -\ln{\frac{1}{1+e^\eta}} \\
		&= \ln(1+e^\eta)
	\end{align}

	\item 高斯分布 \\
	在证明使用最小二乘法作为线性回归的Cost Fuction\footnote{最大化$\theta$等同于最大化$\frac{1}{2} \sum_{i=1}^{m}\left(y^{(i)}-\theta^Tx^{(i)}-\mu\right)^2$}的可行性时提到: $J(\theta)$取到最值时的$\theta$值与$\sigma^2$无关(但与$\mu$有关),故在下面的证明中,我们另$\sigma^2=1$,以简化计算
	\begin{align}
		P(y;\mu) &= \frac{1}{\sqrt{2\pi}}e^{-\frac{1}{2}\left(y-\mu\right)^2} \\
		&= \frac{1}{\sqrt{2\pi}} e^{-\frac{y^2-2\mu y +\mu^2}{2}} \\
		&= \frac{1}{\sqrt{2\pi}}e^{-\frac{y^2}{2}} \cdot e^{\mu y-\frac{\mu^2}{2}}
	\end{align}
	故:
	\begin{align}
		b(y) &= \frac{1}{\sqrt{2\pi}}e^{-\frac{y^2}{2}} \\
		\eta^T &= \mu = \eta \\
		T(y) &= y \\
		a(\eta) &= -1\cdot (-\frac{\mu^2}{2}) = \frac{\mu^2}{2} = \frac{\eta^2}{2}
	\end{align}

	\item 注意事项
	\begin{enumerate}
		\item $\eta^T$要在$\eta$为标量时才能直接等于$\eta$,故上面的式子中是在得到$\eta^T$为标量后再写$=\eta$
		\item $a(\eta)$通常不会直接得到关于$\eta$的函数,需要通过$\eta^T$来与$\eta$关联,从而得到$a(\eta)$
	\end{enumerate}
\end{enumerate}

\subsection{如何构造广义线性模型}
\subsubsection{三个假设}
\begin{enumerate}
	\item 假设一:假设$y|x; \theta$服从指数分布族,即$y|x; \theta \sim E(\eta)$
	\item 假设二:假设$h_\theta(x)$ 是$p\left(T(y)|x; \theta\right)$的无偏估计量,即$h_\theta(x)=E\left(T(y)|x; \theta\right)$,并另$T(y)=y$,于是就有
	\begin{equation}
		h(x) = E(y|x)
	\end{equation}
	{\color{red}{暂时对此假设不怎么理解,待后续理解后补充}}
	\item 假设三:假设$\eta = \theta^Tx$ \\
	当$\theta$是一个向量时,$\eta_i = \theta_i^Tx$
	\item 解释说明
	\begin{enumerate}
		\item $y|x; \theta$表示的是一个事件,就像事件$X$,事件$Y$,只是写法比较不一样而已
		\item 在构造广义线性模型中,我们的目的是通过输入的$x$来得到$\theta$,进而得到$y$;因为$T(y)$是$\eta$的充分统计量,因此,我们通过得到$T(y)$的信息来了解$\eta$的信息{\color{red}{(此条正确性待查)}}
	\end{enumerate}
\end{enumerate}

\subsubsection{普通的最小二乘法}
对于普通的最小二乘法,由前面的证明我们知道$\mu = \eta$,故
\begin{equation}\begin{aligned}
	h_\theta(x) &= E(y|x;\theta) \\
	&= \mu \\
	&= \eta \\
	&= \theta^T x
\end{aligned}\end{equation}

\subsubsection{逻辑回归}
对于逻辑回归,由前面的证明可知$\phi=\frac{e^{\eta}}{1+e^{\eta}} = \frac{1}{1+e^{-\eta}}$,故
\begin{equation}\begin{aligned}
	h_\theta(x) &= E(y|x; \theta) \\
	&= \phi \\
	&= \frac{1}{1+e^{-\eta}} \\
	&= \frac{1}{1+e^{-\theta^Tx}}
\end{aligned}\end{equation}


\subsubsection{说明}
\begin{enumerate}
	\item 函数$g(\eta) = E\left[E(T(y);\eta)\right]$将$T(y)$的期望与自然参数$\eta$联系起来
	\item 我们将$g(\eta)$称为正则响应函数(Canonical Response Function)
	\item 将$g(\eta)$的反函数$g^{-1}(y)$称为正则关联函数(Canonical Link Function)
	\item {\color{red}{关于正则响应函数,正则关联函数的理解待补充}}
\end{enumerate}



% 

















\subsection{思考}
{\color{red}{以下内容均为个人的思考,不一定是正确的。}}
\begin{enumerate}
	\item 机器学习主要就是围绕几个部分进行各种改进:
	\begin{enumerate}
		\item 决策函数(假设函数):$h_\theta(x)$
		\item 最优化方式:
		\item Cost Function: $J(\theta)$
			\begin{enumerate}
				\item Cost Function仅仅只是一种评价预测值与实际值的方式,就算换种评价方式,差距大的仍旧差距大。表面上看起来通过改进Cost Function并没法得到多少改进,但实际上并非如此
				\item 一方面,通过改进Cost Function,可以在计算预测值与实际值误差时给不同的数据点$(x^{(i)}, y^{(i)})$不同的权重,让较重要的点的有较大权重,较不重要的点权重较小。事实上,局部加权回归就是采用此方式
				\item 另一方面,计算Cost Fuction也是需要计算量的,若采用更简单的计算方式也算对算法的一种优化。{\color{gray}{暂未找到实例}}
			\end{enumerate}
		\item 获取最优解的方式,如梯度下降中使用$\theta_j := \theta_j + \frac{\partial J(\theta)}{\theta_j}$进行迭代
	\end{enumerate}

	\item 线性回归

	\item 逻辑回归

	\item 生成学习算法
	\begin{enumerate}
		\item 随机变量连续时,使用高斯判别分析
		\item 随机变量离散时,使用朴素贝叶斯(拉普拉斯平滑)
	\end{enumerate}


	\item 支持向量机闪光点总结:
	\begin{enumerate}
		\item KKT对偶优化条件$\alpha_i^*g_i(w^*) = 0, \quad i=1, \dots, k$:只需要计算支持向量那一部分数据
		\item 特征映射$\phi(\cdot)$:把低维映射到高维,让低维下非线性可分的数据在高维上线性可分
		\item 核函数$K(x,z)$:让特征映射后的内积容易计算
		\item SMO: 
	\end{enumerate}

\end{enumerate}

















\subsection{贝叶斯统计与正则化}
在前面的逻辑回归讨论中,我们通过最大化似然函数来得到$\theta_{ML}$
\begin{align}
	\theta_{ML} = \arg \max_{\theta} \prod_{i=1}^{m}p(y^{(i)}|x^{(i)}; \theta)
\end{align}
在频率学派的观点中,$\theta$是一个参数,是一个定值(而不是随机的),只是我们暂时不知道它的大小是多少,我们的目的就是通过各种办法来预估$\theta$ \\
在贝叶斯学派的观点中,$\theta$被当做是随机变量(同样,他的值是未知的)。此时,我们称$p(\theta)$为$\theta$的先验分布。给定训练集$S=\{(x^{(i)}, y^{(i)})\}_{i=1}^{m}$,可得$\theta$的后验分布为:
\begin{align}
	p(\theta | S) &= \frac{p(X|\theta)p(\theta)}{p(S)} \\
	&= \frac{\left[\prod_{i=1}^{m}p(y^{(i)}|x^{(i)}, \theta) p(\theta)\right]}{\int_{\theta}\left[\prod_{i=1}^{m}p(y^{(i)}|x^{(i)}, \theta) p(\theta)\right]d\theta}
\end{align}
在上面的式子中,$p(y^{(i)}|x^{(i)}, \theta)$取决于你用什么算法,如用逻辑回归时,$p(y^{(i)}|x^{(i)}, \theta)=h_{\theta}(x^{(i)})^{y^{(i)}}\left[1-h_{\theta}(x^{(i)})\right]^{\left[1-y^{(i)}\right]}$\footnote{注意,此时,我们将$\theta$当做随机变量,因此,$\theta$与$x$间的分隔符号应为逗号,不是冒号了},其中,$h_{\theta}(x^{(i)})=\frac{1}{e^{-\theta^Tx^{(i)}}}$ \\
当我们要通过测试数据$x$进行预测时
\begin{align}
	p(y|x, S) = \int_{\theta}p(y|x, \theta)p(\theta|S)d\theta
\end{align}
期望:
\begin{align}
	E(y|x, S) = \int_{y}yp(y|x, S)dy
\end{align}
但是,因为在计算$p(\theta | S)$时需要耗费太多的性能,因此,我们并不会直接进行计算,而仅仅估计$p(\theta | S)$的值,称为最大化后验概率MAP(Maximum A Posteriori)
\begin{align}
	\theta_{MAP} = \arg \max_{\theta} \prod_{i=1}^{m}p(y^{(i)}|x^{(i)}, \theta)p(\theta)
\end{align}
有上式可以发现,最终的结果与最大化似然得到的一致。\\



























































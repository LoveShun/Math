\subsection{最优间隔分类器-优化目标}
\subsubsection{将难以优化的目标转为容易优化的}
\begin{enumerate}
	\item 通过前面的讲述我们知道,为了对给定的数据集进行分类,我们需要找到使得几何间隔最大化的决策边界(在下面的讲述中,先假设我们会遇到的数据集都是线性可切分的,等我们讲到核函数时再去除这个假设)。于是,可以用以下的式子来描述我们要优化的目标
	\begin{align}
		\max_{\gamma, w, b} \gamma
	\end{align}
	根据前面的定义:$\gamma = \min_{i=1,\dots,m}\gamma^{(i)}$,$\gamma$是所有数据点中几何间隔最小值,所以,对于任意数据点,其几何间隔均应大于$\gamma$\footnote{这是约束条件,请注意优化目标与约束条件的差别。}:
	\begin{align}
		y^{(i)}\frac{(w^Tx^{(i)}+b)}{\|w\|} \geq \gamma, \quad i=1, 2, \dots, m
	\end{align}
	因为对于几何间隔,缩放$w$或$b$或都缩放不会影响决策边界的位置\footnote{决策边界始终由$w^Tx^{(i)}+b=0$确定},于是,我们定义$\|w\|=1$,让几何间隔与函数间隔相等,于是:
	\begin{align}
		&\text{优化目标:} \\
		& \qquad \max_{\gamma, w, b} \gamma \\
		&\text{约束条件:} \\
		& \qquad y^{(i)}(w^Tx^{(i)}+b) \geq \gamma, \quad i=1, 2, \dots, m \\
		& \qquad \|w\| = 1
	\end{align}

	\item 经过上面的转化后,$\|w\|=1$这个约束还是不好处理,我们需要想办法去除这个约束,于是将优化目标$\gamma$改写为$\frac{\hat{\gamma}}{\|w\|}$\footnote{课上有人问为什么要使用函数间隔?为什么要对它除以$\|w\|$?实际上$\frac{\hat{\gamma}}{\|w\|}$只是$\gamma$的另一种表述,为了后续消去不好处理的约束$\|w\| = 1$才这样处理的。}:
	\begin{align}
		\max_{\gamma, w, b} \frac{\hat{\gamma}}{\|w\|}
	\end{align}
	约束条件变为:
	\begin{align}
		y^{(i)}\frac{(w^Tx^{(i)}+b)}{\|w\|} \geq \frac{\hat{\gamma}}{\|w\|}, \quad i=1, 2, \dots, m
	\end{align}
	消去$\|w\|$,得:
	\begin{align}
		&\text{优化目标:} \\
		& \qquad \max_{\gamma, w, b} \frac{\hat{\gamma}}{\|w\|} \\
		&\text{约束条件:} \\
		& \qquad y^{(i)}(w^Tx^{(i)}+b) \geq \hat{\gamma}, \quad i=1, 2, \dots, m
	\end{align}
	如上,经过这样的转化,我们将$\|w\|$消掉了,同时$\|w\| = 1$的假设也可以取消了。

	\item 我们令$\hat{\gamma}=1$\footnote{这样做是合理的,因为不论$\hat{\gamma}$取何值,我们都可以通过缩放$\|w\|$来消除实际的$\hat{\gamma}$与$\hat{\gamma}=1$的差距}\footnote{有的地方会将两条经过支持向量的直线标记为$w^Tx+b=1$和$w^Tx+b=-1$,这是正确的,因为,函数间隔中,$y^{(i)}$的作用只是改变函数间隔$\hat{\gamma}$的正负号,而支持向量对应的就是函数间隔为1的点,代入后就可以得到这两条直线的方程},于是优化目标变成:
	\begin{align}
		\max_{\gamma, w, b} \frac{1}{\|w\|}
	\end{align}
	最大化$\frac{1}{\|w\|}$也即最小化$\|w\|$,即最小化$\|w\|^2$,故:
	\begin{align}
		&\text{优化目标:} \\
		& \qquad \min_{\gamma, w, b} \frac{1}{2}\|w\|^2 \\
		&\text{约束条件:} \\
		& \qquad y^{(i)}(w^Tx^{(i)}+b) \geq 1, \quad i=1, 2, \dots, m
	\end{align}
\end{enumerate}




































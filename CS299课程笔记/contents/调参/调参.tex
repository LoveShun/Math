\section{调参}

\subsection{Error Analysis}
\begin{enumerate}
	\item 0-1错分率 or 误分类率
	1. 将当前的数据分成2份,$70\%$作为训练集,$30\%$作为测试集\\
	2. 用新的训练集训练,用新的测试集检验效果\\
	3. 若$J_{train}(\theta)$很小,$J_{test}(\theta)$很大,则说明出现了过拟合
	\item 训练集 \& 交叉验证集 \& 测试集
	\item 过拟合、欠拟合的判断方法
	1. 欠拟合对应高偏差,表现为$J_{cv}(\theta) \approx J_{train}(\theta)$,且二者都很高 \\
	2. 过拟合对应高方差,表现为$J_{cv}(\theta) >> J_{train}(\theta)$,且$J_{train}(\theta)$很小 \\
	3. $J_{taain}(\theta)$对应训练集的学习能力;$J_{cv}(\theta)$对应训练结果对新样本的适应能力,适应能力越强,$J_{cv}(\theta)$越小 \\
	4. 在训练集和验证集(测试集)效果均不好,说明欠拟合;在训练集效果很好,但在说明欠拟合,但在验证集(测试集)效果不好,说明过拟合
\end{enumerate}

\subsection{Error Metrics for Skewed Classes}
\begin{enumerate}
	\item Skewed Classes
	那些两种(或多种)情况发生的概率相关较大的情况称为Skewed Classes。如买彩票中奖的概率与不中奖的概率
	\item True vs. False \& Positive vs. Negative \\ 
	Predicted:1, Actual:1 --- True Positive --- TP; \\
	Predicted:0, Actual:0 --- True Negative ---TN; \\
	Predicted:0, Actual:1 --- False Positive --- FP; \\
	Predicted:1, Actual:0 --- False Negative --- FN; \\
	True \& False 对应预测的是否正确; \\
	Positive \& Negative 对应实际是否发生
	\item 表格表示:横向表示预测,竖向表示实际 \\ 
	\begin{tabular}{|l|c|r|}
	    \hline
	      & 0 & 1 \\ \hline
	    0 & TN & FN \\ \hline
	    1 & FP & TP \\ \hline
	\end{tabular}
	\item Accuracy \& Precision \& Recall
	Accuracy: 发生的概率:$\frac{True}{All} = \frac{TP+TN}{TP+TN+FP+FN}$;  \\
	Precision: 查准率,在预测为1的情况下,实际为1的概率:$\frac{TP}{TP+FP}$ \\
	Recall:召回率,在实际为1的情况下,被预测出来的概率: $\frac{TP}{TP+FN}$
	\item Precision \& Recall均是越高越好,但实际上,两者无法同时都很高。(PS:两者加起来并不一定会等于1,甚至很小情况下才全等于1))
\end{enumerate}

\subsection{如何评价Precision与Recall}
\begin{enumerate}
	\item 使用$F_1 Score = 2\frac{PR}{P+R}$
	\item 用交叉验证集的$F_1$值来选取最大的$F_1$值对应的P和R,不用训练集(或测试集)中的。
	\item 训练集用来训练$\theta$,交叉难集用来选取$\theta$,测试集用来看选取的$\theta$是否可行
\end{enumerate}



\subsection{拟合效果不好时的解决方法指导}
\begin{enumerate}
	\item 获取更多数据 ---- 解决高方差
	\item 减少特征 ---- 解决高方差
	\item 增加特征 ---- 解决高偏差
	\item 增加高阶多项式 ---- 解决高偏差
	\item 减小$\lambda$ ---- 解决高偏差
	\item 增大$\lambda$ ---- 解决高方差
\end{enumerate}

\subsection{不同神经网络的优缺点}
\begin{enumerate}
	\item 小型神经网络 -- 更少的参数;容易出现欠拟合
	\item 大型神经网络 -- 更多的参数;容易出现过拟合
	\item parameters越复杂,或隐藏的层越多,对训练集的拟合效果越好,但若对验证集的拟合效果不好,说明已经过拟合,此时再增加神经网络的复杂度并不能提高神经网络的效果
\end{enumerate}

\subsection{绘制Learning Curve}
绘制Learning Curve时,对训练集计算训练误差时,每次迭代只能使用训练集的部分数据(第i次迭代使用第1到第i个数据);但对验证集计算验证误差时,每次均应使用所有数据









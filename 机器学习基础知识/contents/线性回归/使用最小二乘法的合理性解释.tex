\subsection{使用最小二乘法的合理性解释}
\subsubsection{证明过程}
\begin{enumerate}
	\item 我们可以将实际的$y^{(i)}$分为两个部分:一部分为被我们的计算模型所包括,为$\theta^Tx^{(i)}$;一部分未被我们的计算模型所包括,将其记为$\epsilon^{(i)}$,可得:
	\begin{equation}
		y^{(i)} = \theta^Tx^{(i)} + \epsilon^{(i)}
	\end{equation}

	\item 根据中心极限定理\footnote{中心极限定理大意:许多独立随机变量之和趋向于服从高斯分布。 更详细的待补充...},同时根据经验,我们可以假设$\epsilon^{(i)}$服从高斯分布$N(\mu, \sigma^2)$,故
	\begin{align}
		P(\epsilon^{(i)}) &= \frac{1}{\sqrt{2\pi}\sigma}e^{-\frac{\left(\epsilon^{(i)}-\mu\right)^2}{2\sigma^2}}  \\
		&\downarrow \\
		P(y^{(i)}|x^{(i)};\theta) &= \frac{1}{\sqrt{2\pi}\sigma}e^{-\frac{\left(y^{(i)}-\theta^Tx^{(i)}-\mu\right)^2}{2\sigma^2}}
	\end{align}

	\item 写出其似然函数
	\begin{align}
		L(\theta) &= L(\theta; X, \vec{y}) = P(\vec{y}|X; \theta) \\
		&= \prod_{i=1}^{m}P(y^{(i)}|x^{(i)}; \theta) \\
		&= \prod_{i=1}^{m}\frac{1}{\sqrt{2\pi}\sigma}e^{-\frac{\left(y^{(i)}-\theta^Tx^{(i)}-\mu\right)^2}{2\sigma^2}} \\
		&= \left(\frac{1}{\sqrt{2\pi}\sigma}\right)^m\cdot e^{\sum_{i=1}^{m}{-\frac{\left(y^{(i)}-\theta^Tx^{(i)}-\mu\right)^2}{2\sigma^2}}}
	\end{align}

	\item 显然,求其对数后更好分析
	\begin{align}
		l(\theta) &= \ln{L(\theta)} \\
		&= \ln\left[\left(\frac{1}{\sqrt{2\pi}\sigma}\right)^m\cdot e^{\sum_{i=1}^{m}{-\frac{\left(y^{(i)}-\theta^Tx^{(i)}-\mu\right)^2}{2\sigma^2}}}\right] \\
		&= m\cdot\ln\frac{1}{\sqrt{2\pi}\sigma} - \sum_{i=1}^{m}\frac{\left(y^{(i)}-\theta^Tx^{(i)}-\mu\right)^2}{2\sigma^2} \\
		&= m\cdot\ln\frac{1}{\sqrt{2\pi}\sigma} - \frac{1}{{\sigma^2}}\cdot \frac{1}{2} \sum_{i=1}^{m}\left(y^{(i)}-\theta^Tx^{(i)}-\mu\right)^2
	\end{align}

	\item 由上式可知,若要最小化似然函数$l(\theta)$,等价于最小化$\frac{1}{2} \sum_{i=1}^{m}\left(y^{(i)}-\theta^Tx^{(i)}-\mu\right)^2$,此式子同样是最小二乘法的形式。另常数项$\mu = 0$便可得到前述线性回归Cost Function使用的最小二乘法,故在线性回归中使用最小二乘法计算$J(\theta)$是合理的

\end{enumerate}

\subsubsection{说明}
\begin{enumerate}
	\item 从上述结果中,我们可以发下,让$l(\theta)$取到最值时的$\theta$值与$\sigma^2$无关。后续说明指数分布族\&一般线性模型时将会用到此性质
\end{enumerate}














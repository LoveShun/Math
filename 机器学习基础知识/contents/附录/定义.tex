\section{附录}
\subsection{定义}
\subsubsection{各类变量}
\begin{enumerate}
	\item $i$: 第$i$个数据集,从1开始,截止于m
	\item $j$: 第$j$维特征,从0开始,截止于n
	\item $m$: 数据集大小,共$m$组数据
	\item $n$: 维度大小,共$n$维特征
	\item $i, j, m, n$的关系: $\sum_{i=1}^{m}\sum_{j=0}^{j=n}{\theta_{j}x_j^{(i)}}$
	\item $(x^{(i)}, y^{(j)})$: 第i个训练集
	\item $P(y|x;\theta)$中,$y$做为一个个体;$x:\theta$作为一个个体,表示$x$以$\theta$为参数
\end{enumerate}

\subsubsection{充分统计量}
\begin{enumerate}
	\item 不损失信息的统计量就是充分统计量,它概况了样本中所包含的未知参数的全部信息。
	\item 定义: 设$x_1, x_2, \dots, x_n$是来自某个总体$X$的样本,总体的分布函数为$F(x;\theta)$,若统计量$T=T(x_1, x_2, \dots, x_n)$为$\theta$的充分统计量,则在给定了$T$的取值后,$x_1, x_2, \dots, x_n$的条件分布$P(X_1=x_1, X_2=x_2, \dots, X_n=x_n$与$\theta$无关
	\item 充分统计量不唯一。实际上,样本本身就是参数的一个充分统计量
	\item 因子分解定理{\color{red}{(没看懂)}}: 设总体的概率函数为$p(x;\theta)$,$x_1, x_2, \dots, x_n$是来自总体的样本,则统计量$T=T(x_1, x_2, \dots, x_n)$为充分统计量的充要条件为:
	\begin{equation}
		p(x_1, x_2, \dots, x_n;\theta) = g\left[T(x_1, x_2, \dots, x_n),\theta\right]\cdot h(x_1, x_2, \dots, x_n)
	\end{equation}
	其中,$g(t,\theta)$是通过统计量T的取值而依赖于样本的,而$h(x_1, x_2, \dots, x_n)$不依赖于$\theta$
	\item 更详细内容: http://wenku.baidu.com/view/8813423343323968011c92ed.html
\end{enumerate}

\subsubsection{自然参数}
\begin{enumerate}
	\item 
	\item 
\end{enumerate}


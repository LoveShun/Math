\section{多维随机变量及其分布}
\subsection{二维随机变量及其分布}
\subsubsection{连续型}
\begin{enumerate}
	\item 二维随机变量的分布
	\begin{equation}
		F(x, y) = P(X\leq x, Y \leq y), \quad -\infty < x < +\infty, -\infty < y < + \infty
	\end{equation}

	\item 二维随机变量的边缘分布
	\begin{equation}\begin{aligned}
			F_X(x) &= P(X\leq x) = P(X\leq x, y< +\infty) = F(x, +\infty) \\
			F_Y(y) &= P(Y\leq y) = P(X< +\infty, Y\leq y) = F(+\infty, y)
	\end{aligned}\end{equation}
	\item 二维随机变量的条件分布
	\begin{align}
		F_{X|Y}(x|y) = P(X \leq x | Y=y) 
		= \lim_{\epsilon \to 0^+}P(X \leq x | y - \epsilon < Y \leq y + \epsilon) \\
		= \lim_{\epsilon \to 0^+} \frac{P(X \leq x, y - \epsilon < Y \leq y + \epsilon)}{P(y - \epsilon < Y \leq y + \epsilon)}
	\end{align}
	$F_{Y|X}(y|x)$同理 \\
	若$f(x,y)$在$(x,y)$点连续,且$f_Y(y)>0$,则:
	\begin{equation}
		F_{X|Y}(x|y) = \int_{-\infty}^{x}\frac{f(s,y)}{f_Y(y)}ds
	\end{equation}
	$F_{Y|X}(y|x)=\int_{-\infty}^{y}\frac{f(x,s)}{f_X(x)}ds$

\end{enumerate}

\subsubsection{离散型}
\begin{enumerate}
	\item 二维离散型随机变量的概率分布(或称分布律)
	\begin{equation}
		P(X=x_i, Y=y_j) = p_{ij}, \quad i,j = 1, 2, \dots
	\end{equation}

	\item 二维离散型随机变量的边缘分布
	\begin{align}
		p_{i\cdot} &= P(X=x_i) 
		= \sum_{j=1}^{+\infty}P(X=x_i, Y=y_j) 
		= \sum_{j=1}^{+\infty}p_{ij}, \quad i = 1, 2, \dots \\
		p_{\cdot j} &= P(Y=y_j) 
		= \sum_{i=1}^{+\infty}P(X=x_i, Y=y_j) 
		= \sum_{i=1}^{+\infty}p_{ij}, \quad j = 1, 2, \dots 
	\end{align}
	\item 二维离散型随机变量的条件分布
	\begin{equation}\begin{aligned}
		P(X=x_i|Y=y_j) = \frac{P(X=x_i, Y=y_j)}{P(Y=y_j)} = \frac{p_{ij}}{p_{\cdot j}}, \quad i = 1, 2, \dots
	\end{aligned}\end{equation}


\end{enumerate}

\subsection{概率密度}
\subsubsection{连续型}
\begin{enumerate}
	\item 二维随机变量及其概率密度 \\
	满足以下条件的$f(x, y)$成为$(X, Y)$的概率密度 \\
	\begin{equation}
		F(x, y) = \int_{-\infty}^{x} \int_{-\infty}^{y}f(u, v)dudv, \quad -\infty <x, y < +\infty
	\end{equation}
	\item 边缘密度
	\begin{equation}
		f_X(x) = \int_{-\infty}^{+\infty}f(x, y)dy
	\end{equation}
	\begin{equation}
		f_Y(y) = \int_{-\infty}^{+\infty}f(x, y)dx
	\end{equation}
	\item 条件密度
	\begin{equation}
		f_{X|Y}(x|y) = \frac{f(x,y)}{F_Y(y)}, \quad f_Y(y) > 0
	\end{equation}
\end{enumerate}

\subsection{离散型}
无


\subsection{分布函数与概率密度的性质} % (fold)
\label{sub:性质}
\subsubsection{$F(x,y)$的性质}
\begin{enumerate}
	\item
	\begin{equation}
		F(-\infty, y) = F(x, -\infty) = F(-\infty, \infty) = 0
	\end{equation}
	\begin{equation}
		F(+\infty, +\infty) = 1
	\end{equation}

	\item 
	\begin{equation}
		P(a<X\leq b, c<Y\leq d) = F(b,d) - F(b,c) - F(a,d) + F(a,c)
	\end{equation}
\end{enumerate}

\subsubsection{$P(X=x_i, Y=y_j)=p_{ij}$的性质}
\begin{enumerate}
	\item 
	\begin{equation}
		p_{ij} \geq 0, \quad i,j = 1,2, \dots
	\end{equation}

	\item 
	\begin{equation}
		\sum_i \sum_j p_{ij} = 1
	\end{equation}
\end{enumerate}

\subsubsection{$f(x,y)$的性质}
\begin{enumerate}
	\item 
	\begin{equation}
		f(x, y) \geq 0
	\end{equation}
	\item 
	\begin{equation}
		\int_{-\infty}^{+\infty} \int_{-\infty}^{+\infty}f(x,y)dxdy = 1
	\end{equation}
	\item 
	\begin{equation}
		P((X,y)\in D) = \iint_Df(x,y) dxdy
	\end{equation}
	
\end{enumerate}


\subsection{随机变量的独立性}
\subsubsection{随机变量$X$与$Y$相互独立的充要条件}
\begin{enumerate}
	\item 离散型
	\begin{equation}
		P(X=x_i, Y=y_j) = P(X=x_i)P(Y=y_j)
	\end{equation}
	即
	\begin{equation}
		p_{ij} = p_{i\cdot} p_{\cdot j}
	\end{equation}

	\item 连续型
	\begin{equation}
		f(x, y) = f_X(x)f_Y(y)
	\end{equation}
	
\end{enumerate}


\subsection{二维均匀分布\&二维正态分布}
\begin{enumerate}
	\item 二维均匀分布
		\[ f(x, y)=\begin{cases}
			\frac{1}{A}, \quad (x, y) \in G \\
			0, \quad others
		\end{cases} \]

	\item 二维正态分布
	\begin{equation}
		f(x,y) = \frac{1}{2\pi \sigma_1 \sigma_2 \sqrt{1-\rho^2}}e^{-\frac{1}{2(1-\rho^2)}\left[ \frac{(x-\mu_1)^2}{\sigma_1^2} - \frac{2\rho(x-\mu_1)(y-\mu_2)}{\sigma_1 \sigma_2} + \frac{(y-\mu_2)^2}{\sigma_2^2} \right]}, \quad -\infty<x<+\infty, -\infty<y<+\infty
	\end{equation}
	其中,$\mu_1, \mu_2 > 0, \sigma_1>0, \sigma_2 > 0, -1<\rho < 1$,且均为常数,记作
	\begin{equation}
		(X,Y) \sim N(\mu_1, \mu_2; \sigma_1^2, \sigma_2^2; \rho)
	\end{equation}

	\item 若$(X,Y)$二维正态,则$X, Y$均正态,即: 若$(X, Y) \sim N(\mu_1, \mu_2; \sigma_1^2, \sigma_2^2; \rho)$,则$X \sim N(\mu_1, \sigma_1^2), Y \sim N(\mu_2, \sigma_2^2)$
	
	\item $X, Y$相互独立的充要条件为:$\rho = 0$

	\item 如果$X_1, X_2, \dots X_n$相互独立,且$X_i \sim N(\mu_i, \sigma_i^2) \quad (i = 1, 2, \dots, n)$,则$\sum_{i=1}^{n}C_iX_i \sim N(\sum_{i=1}^{n}C_i\mu_i, \sum_{j=1}^{n}C_j^2\sigma_j^2)$
\end{enumerate}



\subsection{两个随机变量函数$Z=g(X,Y)$的分布}
\begin{enumerate}
	\item $X,Y$均为离散型 \\
	与一维的一致
	\item $X,Y$均为连续型 \\ 
	\begin{enumerate}
		\item 一般情况
		\begin{equation}
			F_Z(z) = P(Z \leq z) = P(g(X,Y) \leq z) = \iint_{g(x,y)\leq z} f(x,y)dxdy
		\end{equation}
		\item 特殊地,若$Z=X+Y$
		若$Z=X+Y$,则
		\begin{align}
			F_Z(z) &= P(X+Y\leq z) = \iint_{x+y \leq z} f(x,y)dxdy \\
			&= \int_{-\infty}^{+\infty}dx\int_{-\infty}^{z-x}f(x,y)dy \\
			&= \int_{-\infty}^{+\infty}dy\int_{-\infty}^{z-y}f(x,y)dx
		\end{align}
		由上可得,$Z=X+Y$的概率密度为:
		\begin{align}
			f_Z(z) &= \int_{-\infty}^{+\infty}f(x, z-x)dx \\
			&= \int_{-\infty}^{+\infty}f(z-y, y)dy
		\end{align}
		特别地,当$X, Y$相互独立时,$f(x,y) = f_X(x)f_Y(y)$,故
		\begin{align}
			f_Z(z) &= \int_{-\infty}^{+\infty}f_X(x)f_Y(z-x)dx \\
			&= \int_{-\infty}^{+\infty} f_X(z-y)f_Y(y)dy
		\end{align}
		这就是卷积公式,记为$f_X * f_Y$
	\end{enumerate}
	\item $X$为离散型,$Y$为连续型时
	\begin{align}
		F_Z(z) &= P(Z\leq z) = P(g(X,Y)\leq z) \\
		&= \sum_iP(X=x_i)P(g(X,Y)\leq z|X=x_i) \\
		&= \sum_ip_iP(g(x_i, Y)\leq z | X=x_i)
	\end{align}
\end{enumerate}














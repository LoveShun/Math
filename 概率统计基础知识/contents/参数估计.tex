\section{参数估计}
\subsection{点估计}
\begin{enumerate}
	\item 点估计 
	\footnote{为了区分估计量与实际值,我们给估计量带个帽子,如:$\hat{\theta}$} \\
	用样本$X_1, X_2, \dots, X_n$构造的统计量$\hat{\theta}(X_1, X_2, \dots, X_n)$来估计未知参数$\theta$称为点估计,统计量$\hat{\theta}(X_1, X_2, \dots, X_n)$称为估计量
	\item 无偏估计量 \\
	若$\hat{\theta}$是$\theta$的估计量,如果$E(\hat{\theta})=\theta$,则称$\hat{\theta} = \hat{\theta}(X_1, X_2, \dots, X_n)$是未知参数$\theta$的无偏估计量
	\item 更有效估计量 \\
	若$\hat{\theta_1}$和$\hat{\theta_2}$都是$\theta$的无偏估计量,且$D(\hat{\theta_1})\leq D(\hat{\theta_2})$,则称$\hat{\theta_1}$比$\hat{\theta_2}$更有效,或$\hat{\theta_1}$比$\hat{\theta_2}$更有效估计量
	\item 一致估计量 \\
	若$\hat{\theta}(X_1, X_2, \dots, X_n)$是$\theta$的估计值,如果$\hat{\theta}$依概率收敛于$\theta$,则称$\hat{\theta}(X_1, X_2, \dots, X_n)$为$\theta$的一致估计量
	\item 估计量\&统计量 \\
	估计量来自于样本,是一个值;\\
	统计量同样来自样本,但它是一个关于样本的,不含未知参数的函数。
\end{enumerate}

\subsection{矩估计法}
\begin{enumerate}
	\item 矩估计法 \\
	用样本矩估计相应的总体矩,用样本矩的函数估计总体矩相应的函数,然后求出要估计的参数,称这种估计法为矩估计法
	\item 矩估计法的步骤 \\
	略
\end{enumerate}

\subsection{最大似然估计法}
\begin{enumerate}
	\item 似然函数
	\begin{enumerate}
		\item 离散型 \\
		假设其概率分布为:$P(X=a_i) = p(a_i;\theta), \quad i=1,2, \dots$,则似然函数为
		\begin{equation}
			L(\theta) = L(X_1, X_2, \dots, X_n; \theta) = \prod_{i=1}^{n}p(X_i;\theta)
		\end{equation}
		\item 连续型 \\
		假设其概率密度为$f(x;\theta)$,则其似然函数为:
		\begin{equation}
			L(\theta) = L(X_1, X_2, \dots, X_n; \theta) = \prod_{i=1}^{n}f(X_i;\theta)
		\end{equation}
	\end{enumerate}
	\item 最大似然估计法求解步骤
	\begin{enumerate}
		\item 写出似然函数
		\item 求似然函数的导数
		\begin{equation}
			\frac{\mathrm{d}L(\theta)}{\mathrm{d}\theta} = 0
		\end{equation}
		或其对数的导数
		\begin{equation}
			\frac{\mathrm{d}\ln L(\theta)}{\mathrm{d}\theta} = 0
		\end{equation}
		\item 解方程组,得到其驻点
		\item 对每个驻点进行分析,得到最值
		\item 注意:有时使得似然函数达到最值的$\hat{\theta}$不一定是$L(\theta)$或$\ln L(\theta)$的驻点,这种情况下不能用似然方法来求解,应使用其他方法
	\end{enumerate}
\end{enumerate}

\subsection{区间估计}
\begin{enumerate}
	\item 置信区间 \\
	设$\theta$是总体$X$的未知参数,$X_1, X_2, \dots, X_n$是来自总体$X$的样本,对于给定的$\alpha(0<\alpha<1)$,如果两个统计量满足
	\begin{equation}
		P(\theta_1<\theta<\theta_2) = 1-\alpha
	\end{equation}
	则称随机区间$(\theta_1, \theta_2)$为参数$\theta$的置信水平(或置信度)为$1-\alpha$的置信区间(或区间估计),简称为$\theta$的$1-\alpha$置信区间,$\theta_1$和$\theta_2$分别称为置信下限和置信上限
	\item 一个正态总体参数的区间估计 \\
	略
	\item 两个正态总体参数的区间估计 \\
	略
\end{enumerate}
















\section{梯度下降(GD)}
\subsection{假设函数(Hypothesis Function)}
\begin{equation}\begin{aligned}
	h_\theta(x) = \theta_0 + \theta_1x_1 + \theta_2x_2 + \theta_3x_3 + ... + \theta_nx_n
\end{aligned}\end{equation}
另$x_0=1$,则
\begin{equation}\begin{aligned}
	h_\theta(x) &= \theta_0x_0 + \theta_1x_1 + \theta_2x_2 + \theta_3x_3 + ... + \theta_nx_n \\
	&= \sum_{j=0}^n{\theta_jx_j}
\end{aligned}\end{equation}

\subsection{批梯度下降(BGD)}
\begin{enumerate}
	\item Cost Function
	\begin{equation}\begin{aligned}
		J(\theta) = \frac{1}{2} \sum_{i=1}^m \left[h_{\theta} {(x^{(i)})} - y^{(i)}\right]^2
	\end{aligned}\end{equation}
	其中,$J(\theta)$就是在训练过程中的Cost Function。我们的目的就是将该Cost Function最小化。
	在以上式子中:\\
		$x^{(i)}$为训练集的输入;\\
		$y^{(i)}$为训练集的输出;\\
		$\theta$为所要训练的参数;\\
		$h_{\theta}$为假设函数;\\
		$h_{\theta} {(x^{(i)})}$为训练集输入经过假设函数后得到的结果\\
		以上式子中,其实是使用最小二乘法

	\item 迭代方式
	\begin{equation}\begin{aligned}
	      \theta_j &:= \theta_j - \alpha \frac{\partial} {\partial \theta_j} J(\theta)
	\end{aligned}\end{equation}
	其中:
	\begin{equation}\begin{aligned}
	      \frac{\partial} {\partial \theta_j} J(\theta) &= \frac{\partial}{\partial \theta_j} \frac{1}{2} \sum_{i=1}^m\left[ h_\theta(x^{(i)}) - y^{(i)} \right]^2 \\
	      &= 2 * \frac{1}{2} \sum_{i=1}^m\left[ h_\theta(x^{(i)}) - y^{(i)} \right] \frac{\partial}{\partial\theta_j}\left[ h_\theta(x^{(i)}) - y^{(i)} \right] \\
	      &= \sum_{i=1}^m\left[ h_\theta(x^{(i)}) - y^{(i)} \right]\frac{\partial}{\partial\theta_j}\left[ \theta_0x_0^{(i)} +  \theta_1x_1^{(i)} + \theta_2x_2^{(i)} + ... + \theta_nx_n^{(i)} - y^{(i)} \right] \\
	      &= \sum_{i=1}^m\left[ h_\theta(x^{(i)}) - y^{(i)} \right]x_j^{(i)}
	\end{aligned}\end{equation}
	所以,更新后的梯度下降公式为
	\begin{equation}\begin{aligned}
		\theta_j &:= \theta_j - \alpha \sum_{i=1}^m \left[ h_\theta(x^{(i)}) - y^{(i)} \right]x_j^{(i)}
	\end{aligned}\end{equation}
	其中,$\alpha$称为学习速率(learning rate),用于控制$\theta$前进的步伐,避免$\theta$走得太快(或太慢)。\\

	梯度下降法每次计算都是找到当前所在位置中,下降最快的方向(至于为什么是当前所在位置中下降最快的方向就是梯度的定义了)。 \\
	如果计算进入了某个局部极小值,则很可能一直在这个局部极小值内出不来了(除非你的步长比较大,帮助它跑出这个局部极小值),这就是梯度下降法的局限所在,它很容易陷入局部极小值中。 \\
	使用梯度下降法每次修正的时参数$\theta$,而不是输入的$X$或$y$ \\

	Todo PS: 以下内容正确性待思考。因为其是当前位置下,下降最快的方向,若出现每个维度求导后结果都是0,则此时不论再怎么迭代,这个点都不会再变化了(或者其他导致每个维度的偏导数都为0的情况也类似,当然,这种情况较少见)
\end{enumerate}


\subsection{随机梯度下降(SGD)}
由批梯度下降的式子可以发现,在每一次梯度下降的迭代过程中,我们遍历了所有训练集。这将会耗费大量的性能,于是产生了随机梯度下降(SGD)方法,每次迭代只使用一个数据。 \\
\begin{enumerate}
	\item Cost Function
	\begin{equation}\begin{aligned}
		J(\theta) = \frac{1}{2} \left[h_{\theta} {(x^{(i)})} - y^{(i)}\right]^2
	\end{aligned}\end{equation}

	\item 迭代方式
	\begin{equation}\begin{aligned}
	      \frac{\partial} {\partial \theta_j} J(\theta) &= \left[ h_\theta(x^{(i)}) - y^{(i)} \right]x_j^{(i)}
	\end{aligned}\end{equation}
	更新后的梯度下降公式为
	\begin{equation}\begin{aligned}
		\theta_j &:= \theta_j - \alpha\left[ h_\theta(x^{(i)}) - y^{(i)} \right]x_j^{(i)}
	\end{aligned}\end{equation}
	如上所示,与批梯度下降不同,随机梯度下降每次迭代只用了一个数据(第$i$个数据),若$i$从1取到m,则完成了一次训练集的遍历。\\
	使用随机梯度下降虽然解决了批梯度下降耗费过多性能的问题,但是却带来了另一个问题:收敛太慢!由此,我们折中使用迷你批梯度下降(mini-batch GD)。\\
	随机梯度下降(包括其他梯度下降方法)均允许多次遍历所有数据集。
\end{enumerate}

\subsection{迷你批梯度下降(mini-batch GD)}
为了解决梯度下降每次都使用所有训练集导致的性能问题,以及随机梯度下降每次只使用一个数据导致的收敛太慢,我们可以只用mini-batch GD。每次只用一部分训练集进行训练。
\begin{enumerate}
	\item Cost Function \\
	略

	\item 迭代方式 \\
	略
\end{enumerate}











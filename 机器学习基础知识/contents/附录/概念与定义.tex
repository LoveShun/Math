\section{附录}
\subsection{概念与定义}
\subsubsection{各类变量}
\begin{enumerate}
	\item $i$: 第$i$个数据集,从1开始,截止于m
	\item $j$: 第$j$维特征,从0开始,截止于n
	\item $m$: 数据集大小,共$m$组数据
	\item $n$: 维度大小,共$n$维特征
	\item $i, j, m, n$的关系: $\sum_{i=1}^{m}\sum_{j=0}^{j=n}{\theta_{j}x_j^{(i)}}$
	\item $(x^{(i)}, y^{(j)})$: 第i个训练集
	\item $P(y|x;\theta)$中,$y$做为一个个体;$x:\theta$作为一个个体,表示$x$以$\theta$为参数
	\item 先验概率: 事情还没有发生,要求这件事情发生的可能性的大小,是先验概率. 
	\item 后验概率: 事情已经发生,要求这件事情发生的原因是由某个因素引起的可能性的大小,是后验概率.
\end{enumerate}

\subsubsection{充分统计量}
\begin{enumerate}
	\item 不损失信息的统计量就是充分统计量,它概况了样本中所包含的未知参数的全部信息。
	\item 定义: 设$x_1, x_2, \dots, x_n$是来自某个总体$X$的样本,总体的分布函数为$F(x;\theta)$,若统计量$T=T(x_1, x_2, \dots, x_n)$为$\theta$的充分统计量,则在给定了$T$的取值后,$x_1, x_2, \dots, x_n$的条件分布$P(X_1=x_1, X_2=x_2, \dots, X_n=x_n$与$\theta$无关
	\item 充分统计量不唯一。实际上,样本本身就是参数的一个充分统计量
	\item 因子分解定理{\color{red}{(没看懂)}}: 设总体的概率函数为$p(x;\theta)$,$x_1, x_2, \dots, x_n$是来自总体的样本,则统计量$T=T(x_1, x_2, \dots, x_n)$为充分统计量的充要条件为:
	\begin{equation}
		p(x_1, x_2, \dots, x_n;\theta) = g\left[T(x_1, x_2, \dots, x_n),\theta\right]\cdot h(x_1, x_2, \dots, x_n)
	\end{equation}
	其中,$g(t,\theta)$是通过统计量T的取值而依赖于样本的,而$h(x_1, x_2, \dots, x_n)$不依赖于$\theta$
	\item 更详细内容: http://wenku.baidu.com/view/8813423343323968011c92ed.html
\end{enumerate}

\subsubsection{自然参数}
\begin{enumerate}
	\item 
	\item 
\end{enumerate}

\subsubsection{先验概率 \& 后验概率}
\begin{enumerate}
	\item 先验概率: 事情还没有发生,要求这件事情发生的可能性的大小,是先验概率. 
	\item 后验概率: 事情已经发生,要求这件事情发生的原因是由某个因素引起的可能性的大小,是后验概率.
\end{enumerate}

\subsubsection{似然函数}
\begin{enumerate}
	\item 似然函数的定义:给定$x$时,关于参数$\theta$的似然函数$L(\theta)$(在数值上)等于给定$\theta$后变量$X=x$的概率:
	\begin{equation}
		L(\theta|x) = P(X=x|\theta)
	\end{equation}
	\item 从定义上讲,似然函数和概率密度函数是两个不同的数学对象,前者是关于未知参数$\theta$的函数,后者是关于$x$的函数。两者的相等仅仅是在数值上的相等。
	\item 似然函数$L(\theta|x)$表示的意义是:在参数$\theta$下,随机变量$X$取到$x$的可能性有多大;概率密度$f(x|\theta)$表示的意义:在给定样本$x$时,参数$\theta$使得出现$X=x$的可能性有多大。二者表示的都是可能性。
	\item 若$X$是一个离散型随机变量,则$f(x|\theta)$可改写成$f(x|\theta)=P_\theta(X=x)$。式子$L(\theta_1|x)=P_{\theta_1}(X=x) > L(\theta_2|x)=P_{\theta_2}(X=x)$表示:在参数$\theta_1$下,随机变量$X$取到$x$的可能性大于参数$\theta_2$下随机变量$X$取到$x$的可能性。
	\item 若$X$是一个联系型随机变量也类似。
	\item 参考资料:\href{https://www.zhihu.com/question/54082000}{如何理解似然函数?} \\
	\href{http://baike.baidu.com/subview/1864828/1864828.htm}{似然函数-百度百科}
\end{enumerate}

\subsubsection{参数学习算法 \& 非参数学习算法}
\begin{enumerate}
	\item 参数学习算法: 对于线性回归算法,一旦拟合出适合训练数据的参数$\theta$,保存这些参数$\theta$,对于之后的预测,不需要再使用原始训练数据集; 
	\item 非参数学习算法: 对于局部加权线性回归算法,每次进行预测都需要全部的训练数据(每次进行的预测得到不同的参数$\theta$),没有固定的参数$\theta$
\end{enumerate}












































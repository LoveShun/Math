\section{假设检验}
\subsection{假设检验}
\begin{enumerate}
	\item 实际推断原理: 小概率事件在一次试验中是不会发生的。 \\
	实际推断原理又称小概率原理。
	\item 假设检验
	\begin{enumerate}
		\item 假设是关于总体的论断或命题,常用字母"$H$"表示。
		\item 假设分为基本假设$H_0$和备选假设 \\
		基本假设又称原假设、零假设 \\
		备选假设又称备择假设、对立假设
		\item 假设检验: 根据样本,按照一定规则判断所做假设$H_0$的真伪,并作出接受还是拒绝接受$H_0$的决定。
	\end{enumerate}
	\item 两类错误
	\begin{enumerate}
		\item 拒绝实际真的假设$H_0$(弃真)称为第一类错误
		\item 接受实际不真的假设$H_0$(纳伪)称为第二类错误
	\end{enumerate}
	\item 显著性检验
	\begin{enumerate}
		\item 显著性水平:在假设检验中允许犯第一类错误的概率称为显著水平,记为$\alpha(0<\alpha<1)$。
		它表现了对$H_0$弃真的控制程度,一般$\alpha$取0.1, 0.05, 0.01, 0.001等
		\item 显著性检验: 只控制第一类错误概率$\alpha$的统计检验称为显著性检验
		\item 显著性检验的一般步骤 \\
		略
	\end{enumerate}
\end{enumerate}


\subsection{正态总体参数的假设检验}
略











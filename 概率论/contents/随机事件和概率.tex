\section{随机事件和概率}
\subsection{基本概念及公式}
\begin{enumerate}
	\item 完备事件组 \\
	满足$\cup_{i=1}^nB_i=\Omega$,且$B_iB_j = \emptyset$的事件组$B_1$, $B_2$, ..., $B_n$成为完备事件组
	\item 乘法公式
	\begin{align}
	P(A_1A_2...A_n) &= P(A_1)P(A_2|A_1)P(A_3|A_1A_2)....P(A_n|A_1A_2...A_{n-1})
	\end{align}
	乘法公式并没有$A_k$要相互独立或条件独立的要求。

	\item 全概率公式
	\begin{equation}
	P(A) = \sum_{i=1}^nP(B_i)P(A|B_i)
	\end{equation}
	其中,$B$为一完备事件组

	\item 贝叶斯公式
	\begin{align}
	P(B_j|A) &= \frac{P(B_j)P(A|B_j)}{P(A)} \\
	&= \frac{P(B_j)P(A|B_j)}{\sum_{i=1}^nP(B_i)P(A|B_i)},  \quad j = 1, 2, 3, ..., n
	\end{align}
	\footnote{$P(B_j|A)=\frac{P(AB_j)}{P(A)}$,再将分子用乘法公式展开,分母用全概率公式展开,即可得到贝叶斯公式} \\
	同样,要求B为完备事件组。 \\
	贝叶斯公式求得是条件概率,求解过程中一般需要用到全概率公式(除非$P(A)$已知)。 \\
	事情还没有发生,要求这件事情发生的可能性的大小,是先验概率;事情已经发生,要求这件事情发生的原因是由某个因素引起的可能性的大小,是后验概率。所以贝叶斯公式求的是后验概率。
	\item 其他
	\begin{enumerate}
		\item 加法公式
		\begin{align}
			P(A\cup B) &= P(A) + P(B) - P(AB) \\
			P(A\cup B\cup C) &= P(A)+P(B)+P(C)-P(AB)-P(BC)-P(AC) +P(ABC)
		\end{align}
		\item 减法公式
		\begin{equation}
			P(A-B) = P(A)-P(AB)
		\end{equation}
	\end{enumerate}
\end{enumerate}

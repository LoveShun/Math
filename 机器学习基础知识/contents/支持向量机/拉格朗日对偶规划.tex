\subsection{拉格朗日对偶规划}
\subsubsection{拉格朗日乘数法}
\begin{enumerate}
	\item 在等式约束下求最优问题求解中可使用拉格朗日乘数法,其要解决的问题可表述如下:
	\begin{align}
		&\text{优化目标:} \\
		& \qquad \min_{w} f(w) \\
		&\text{约束条件:} \\
		& \qquad h_i(w) = 0, \quad i=1,\dots,l
	\end{align}
	$l$是等式约束的个数
	\item 其拉格朗日函数为:
	\begin{align}
		\mathcal{L}(w,\beta) = f(w) + \sum_{i=1}^{l}\beta_ih_i(w)
	\end{align}
	\item 求拉格朗日函数的偏导数
	\begin{align}
		\frac{\partial \mathcal{L}}{\partial w_i} &= 0\\
		\frac{\partial \mathcal{L}}{\partial \beta_i} &= 0
	\end{align}
	求解上式,得到$w, \beta$即可
\end{enumerate}

\subsubsection{拉格朗日对偶规划}
\begin{enumerate}
	\item 对于有不等约束的问题,拉格朗日乘数法就无能为力了;这时就需要用拉格朗日对偶规划,其所要解决的问题可表述如下:
	\begin{align}
		&\text{优化目标:} \\
		& \qquad \min_{w} f(w) \\
		&\text{约束条件:} \\
		& \qquad g_i(w) \leq 0, \quad i=1,\dots,k \\
		& \qquad h_i(w) = 0, \quad i=1,\dots,l
	\end{align}
	$k, l$为对应约束的个数。\\
	称其为原问题。
	\item 其Generalized Lagrangian\footnote{{\color{red}{不知道怎么翻译}}}为:
	\begin{align}
		\mathcal{L}(w, \alpha, \beta) = f(w) + \sum_{i=1}^{k}\alpha_ig_i(w) + \sum_{i=1}^{l}\beta_ih_i(w)
	\end{align}
	\item 下面我们定义以下式子:
	\begin{align}
		\theta_{\mathcal{P}}(w) = \max_{\alpha, \beta; \alpha_i\geq0} \mathcal{L}(w, \alpha, \beta)
	\end{align}
	注意,在上面的式子中,我们对$\alpha_i$做了限制:$\alpha_i\geq 0$,于是,$\theta_{\mathcal{P}}(w)$便有了以下性质:
	\begin{enumerate}
		\item 当$g_i(w)$不满足约束条件,即$g_i(w)>0$时,为了使$\mathcal{L}(w, \alpha, \beta)$取到更大,我们只需要让$\alpha_i$更大就行,很显然,此时$\max \mathcal{L}(w, \alpha, \beta) = +\infty = \theta_{\mathcal{P}}(w)$
		\item 当$h_i(w)$不满足约束条件时,即$h_i(w)>0$或$h_i(w)<0$,我们同样只需要更改$\beta_i$的值就能让$\max \mathcal{L}(w, \alpha, \beta)$取到$+\infty$
		\item 当$g_i(w)$与$h_i(w)$均满足约束条件时,$\sum_{i=1}^{k}\alpha_i g_i(w)$的最大值是0;$\sum_{i=1}^{l}\beta_ih_i(w)$始终为0;所以,$\theta_{\mathcal{P}}(w)=\max_{\alpha, \beta; \alpha_i\geq0} \mathcal{L}(w, \alpha, \beta)=f(w)$
		\item 综上,我们可以得到以下式子:
		\[ \theta_{\mathcal{P}}(w)=\begin{cases}
		f(w) \quad \text{符合所有约束条件} \\
		+\infty \quad \text{其他}
		\end{cases} \]
	\end{enumerate}

	\item 通过上面的式子我们可以知道,在满足所有约束条件的情况下,$f(w)=\theta_{\mathcal{P}}(w)$,求解$f(w)$的最小值就等同于求解$\theta_{\mathcal{P}}(w)$的最小值(所以我们将$\theta_{\mathcal{P}}(w)$称为{\color{blue}{原问题}}),于是我们的优化目标就变成了:
	\begin{align}
		\min_{w} \max_{\alpha, \beta; \alpha_i \geq 0} \mathcal{L}(w, \alpha, \beta)
	\end{align}
	注意:
	\begin{enumerate}
		\item 对于$\min_{w}$,这是由我们的优化目标$\min_{w}f(w)$带来的
		\item $\max_{\alpha, \beta; \alpha_i \geq 0} \mathcal{L}(w, \alpha, \beta)$是一个整体,此式子表示的是取得$\mathcal{L}(w, \alpha, \beta)$的最大值的最小值
	\end{enumerate}

	\item 相对于前面的求拉格朗日函数最大值$\theta_{\mathcal{P}}(w) = \max_{\alpha, \beta; \alpha_i\geq0} \mathcal{L}(w, \alpha, \beta)$,其{\color{blue}{对偶问题}}\footnote{为什么要讲到其对偶问题呢?后面会讲到。}为求拉格朗日函数的最小值:
	\begin{align}
		\theta_{\mathcal{D}}(\alpha, \beta) = \min_{w} \mathcal{L}(w, \alpha, \beta)
	\end{align}
	对偶问题对应的{\color{blue}{对偶优化问题}}为:
	\begin{align}
		\max_{\alpha, \beta;\alpha_i \geq 0} \theta_{\mathcal{D}}(\alpha, \beta) = \max_{\alpha, \beta;\alpha_i \geq 0} \min_{w} \mathcal{L}(w, \alpha, \beta)
	\end{align}
	\footnote{注意,这个对偶优化问题并不是我们推导出来的,可以说是我们为了后续需要设计出来的,所以不要去想为什么这个式子是这样子的了。}

	\item 对比一下原问题与其对偶问题,可以发现,两者之间的差别只是对调了求最大值$\max_{\alpha, \beta;\alpha_i \geq 0}$与最小值$\min_{w}$的顺序。在后面的描述中,我们用$p^{*}$表示原问题的最优解,用$d^{*}$表示对偶问题的最优解:
	\begin{align}
		p^{*} &= \min_{w} \max_{\alpha, \beta; \alpha_i \geq 0} \mathcal{L}(w, \alpha, \beta) \\
		d^{*} &= \max_{\alpha, \beta;\alpha_i \geq 0} \min_{w} \mathcal{L}(w, \alpha, \beta)
	\end{align}

	\item 原问题与其对偶问题有以下性质:
	\begin{align}
		d^{*} = \max_{\alpha, \beta;\alpha_i \geq 0} \min_{w} \mathcal{L}(w, \alpha, \beta) \leq 
		\min_{w} \max_{\alpha, \beta; \alpha_i \geq 0} \mathcal{L}(w, \alpha, \beta) = p^{*}
	\end{align}
	在一些特定条件(KKT条件)下,取等号。这时,就可以将原优化问题转化为对偶优化问题求解了。

	\item KKT条件 \\
	假设$f(w), g_i(w)$是凸函数、$h_i(w)$是仿射函数、存在$w$使得$g_i(w) < 0$对所有的$i$都成立,那么一定会存在$w^*, \alpha^*, \beta^*$\footnote{其中$w^*$是原问题的解,$\alpha^*, \beta^*$是对偶问题的解,且$p^*=d^*=\mathcal{L}(w^*, \alpha^*, \beta^*)$}满足以下条件:
	\begin{align}
		\frac{\partial\mathcal{L}(w^*, \alpha^*, \beta^*)}{\partial w_i} &= 0, \quad i=1, \dots, n \\
		\frac{\partial\mathcal{L}(w^*, \alpha^*, \beta^*)}{\partial \beta_i} &= 0, \quad i=1, \dots, l \\
		\alpha_i^*g_i(w^*) &= 0, \quad i=1, \dots, k \\
		g_i(w^*) &\leq 0, \quad i=1, \dots, k  \\
		\alpha^* &\geq 0, \quad i=1, \dots, k 
	\end{align}
	其中:\\
	$\alpha_i^*g_i(w^*) = 0, \quad i=1, \dots, k$称为{\color{blue}{KKT对偶互补条件}};当$\alpha^* > 0$时,$g_i(w^*) = 0$。后续会用到此性质。
\end{enumerate}
本部分内容参考资料:\url{http://blog.csdn.net/Victor_Gun/article/details/45227999}









































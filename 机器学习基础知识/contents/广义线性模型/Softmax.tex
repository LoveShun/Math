\subsection{Softmax Regression}
\subsubsection{关于$\phi$的说明}
\begin{enumerate}
	\item 对于总共有$k$类的分类问题,因为每个分类出现的概率和为1,故仅有$k-1$维特征是相互独立的
	\item 我们将$p(y=i;\phi)$记为$\phi_i$;于是$\phi_k=p(y=k;\phi)=1-\sum_{i=1}^{k-1}\phi_i$。
\end{enumerate}

\subsubsection{关于$T(y)$的表示法说明}
\begin{enumerate}
	\item 为了描述$T(y)$属于多分类问题中的那一类,我们用$k-1$维向量来描述$T(y)$,其中,若$T(y)$属于第$i$类,则第$i$维值为1,其余均为0\footnote{此种表示方法称为One-Hot}。示例如下:
	\begin{equation}
		T(2) = \left[\begin{matrix} 0 \\ 1 \\ 0 \\ \vdots \\ 0 \end{matrix}\right], \quad
		T(1) = \left[\begin{matrix} 1 \\ 0 \\ 0 \\ \vdots \\ 0 \end{matrix}\right], \quad 
		T(k-1) = \left[\begin{matrix} 0 \\ 0 \\ 0 \\ \vdots \\ 1 \end{matrix}\right]
	\end{equation}
	于是,$T(k)$就是个$\vec{0}$向量\footnote{因为它$0\to n-1$维均不是}
	\begin{equation}
		T(k) = \left[\begin{matrix} 0 \\ 0 \\ 0 \\ \vdots \\ 0 \end{matrix}\right]
	\end{equation}
	\item 为了描述方便,我们使用以下新的表示方法
	\begin{enumerate}
		\item $1\{\cdot\}$,在$\cdot$的值为$True$时,$1\{True\}=1$,否则$1\{False\}=0$。如$1\{2=3\}=0$,$1\{2=2\}=1$
		\item 在以上表示法下,$T(y)$可表示成以下形式
		\begin{align}
			\left(T(y)\right)_i &= 1\{y=i\} \\
			& \downarrow \\
			E\left[\left(T(y)\right)_i\right] &= P(y=i)=\phi_i
		\end{align}
	\end{enumerate}
\end{enumerate}

\subsubsection{证明多项式分布属于指数族分布}
\begin{enumerate}
	\item 在多分类问题中,$p(y;\phi)$表示的含义及其表示方式
	\begin{enumerate}
		\item 与逻辑回归时的二分类一样,$p(y=k;\phi)$表示的是取到第$k$类的概率,用$p(y;\phi)$将$p(y=1;\phi), p(y=2;\phi), \dots, p(y=k;\phi)$表示成一个表达式
		\item 参考逻辑回归的方式,我们将$p(y;\phi)$表示为
		\begin{align}
			p(y;\phi) &= \phi_{1}^{1\{y=1\}}\phi_{2}^{1\{y=2\}}\dots\phi_{k}^{1\{y=k\}} \\
			&= \phi_{1}^{1\{y=1\}}\phi_{2}^{1\{y=2\}}\dots\phi_{k}^{1-\sum_{i=1}^{k-1}1\{y=i\}} \\
			&= \phi_{1}^{\left(T(y)\right)_1}\phi_{2}^{\left(T(y)\right)_2}\dots\phi_{k}^{1-\sum_{i=1}^{k-1}\left(T(y)\right)_i} \\
			&= e^{\left(T(y)\right)_1\ln\phi_1 + \left(T(y)\right)_2\ln\phi_2 + \dots + \left(T(y)\right)_{k-1}\ln\phi_{k-1} + \left[1-\sum_{i=1}^{k-1}\left(T(y)\right)_i\right]\ln\phi_k } \\
			&= e^{\left(T(y)\right)_1\ln\frac{\phi_1}{\phi_k} + \left(T(y)\right)_2\ln\frac{\phi_2}{\phi_k} + \dots + \left(T(y)\right)_{k-1}\ln\frac{\phi_{k-1}}{\phi_k} + \ln\phi_k} \\
			&= b(y)e^{\eta^T T(y) - a(\eta)}
		\end{align}
		\end{enumerate}
	\item 从上式可得
	\begin{align}
		b(y) &= 1 \\
		\eta^TT(y) &= \left(T(y)\right)_1\ln\frac{\phi_1}{\phi_k} + \left(T(y)\right)_2\ln\frac{\phi_2}{\phi_k} + \dots + \left(T(y)\right)_{k-1}\ln\frac{\phi_{k-1}}{\phi_k} \\
		&\downarrow \\
		\eta^T &= \left[\begin{matrix}
		\ln\frac{\phi_1}{\phi_k} \\ \ln\frac{\phi_2}{\phi_k} \\ \vdots \\ \ln\frac{\phi_{k-1}}{\phi_k}
		\end{matrix}\right] \\
		a(\eta) &= -\ln\phi_k
	\end{align}
\end{enumerate}


\subsubsection{使用Softmax进行分类}
\begin{enumerate}
	\item 对于$\eta$中的某一项
	\begin{align}
		\eta_i &= \ln{\frac{\phi_i}{\phi_k}} \\
		&\downarrow \\
		e^{\eta_i} &= \phi_i \\
		&\downarrow \\
		\phi_k e^{\eta_i} &= \phi_i \\
		\phi_k\sum_{i=1}^{k}e^{\eta_i} &= \sum_{i=1}^{k}\phi_i = 1 \\
		&\Downarrow \\
		\phi_i &= \phi_k\cdot e^{\eta_i} \\
		&= e^{\eta_i} \cdot \frac{1}{\sum_{j=1}^{k}e^{\eta_j}}
	\end{align}
	如上所示,$\phi_i = \frac{e^{\eta_i}}{\sum_{j=1}^{k}e^{\eta_j}}$称为Softmax函数
	\item 最后,再使用前面的假设三:$\eta_i = \theta_i^Tx, \quad i=1,2, \dots, k-1, \quad \theta_i \in {\rm I\!R}^{n+1}$,可得到
	\begin{align}
		p(y=i|x; \theta) &= \phi_i \\
		&= \frac{e^{\eta_i}}{\sum_{j=1}^{k}e^{\eta_j}} \\ 
		&= \frac{e^{\theta_i^Tx}}{\sum_{j=1}^{k}e^{\theta_j^Tx}}
	\end{align}
	\item 故
	\begin{align}
		h_\theta(x) &= E\left[T(y)|x;\theta\right] \\
		&= E\left[\left[\begin{matrix} 1\{y=1\} \\ 1\{y=2\} \\ \vdots \\ 1\{y=k-1\} \end{matrix}\right|x;\theta\right] \\
		&= \left[\begin{matrix}\phi_1 \\ \phi_2 \\ \vdots \\ \phi_{k-1} \end{matrix}\right] \\ 
		&= \left[\begin{matrix}
		\frac{e^{\theta_1^Tx}}{\sum_{j=1}^{k}e^{\theta_j^Tx}} \\
		\frac{e^{\theta_2^Tx}}{\sum_{j=1}^{k}e^{\theta_j^Tx}} \\
		\vdots \\
		\frac{e^{\theta_{k-1}^Tx}}{\sum_{j=1}^{k}e^{\theta_j^Tx}}
		\end{matrix}\right]
	\end{align}
	如上,最终$h_\theta(x)$可得到$k-1$维向量,通过总概率为1得到第$k$维的概率值,由此可以得到每一种类别的概率值
	\item 以上得到多分类中每种分类概率值的方法就叫做Softmax回归(Softmax Regression)
\end{enumerate}


\subsubsection{参数$\theta$的拟合方式}
使用对数似然估计方法
\begin{align}
	l(\theta) &= \sum_{i=1}^{m}\ln p\left[y^{(i)}|x^{(i)};\theta\right] \\
	&= \sum_{i=1}^{m} \ln \prod_{l=1}^{k}\left[
	\frac{e^{\theta_l^Tx^{(i)}}}{\sum_{j=1}^{k}e^{\theta_l^Tx^{(i)}}}
	\right]^{1\{y^{(i)}=l\}}
\end{align}
{\color{red}{后续待补充}}











